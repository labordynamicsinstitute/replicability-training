\documentclass[xcolor={dvipsnames}]{beamer}
\title[LDI Training]{Walkthrough of Setting Up a Replication Example}
\author{Lars Vilhuber and David Wasser}
\date{May 17, 2019}
\usetheme{Boadilla}
\setbeamertemplate{navigation symbols}{}
\setbeamertemplate{footline}{
  \leavevmode%
  \hbox{%
  \begin{beamercolorbox}[wd=.5\paperwidth,ht=2.25ex,dp=1ex,center]{author in head/foot}%
    \usebeamerfont{author in head/foot}\insertshortauthor
  \end{beamercolorbox}%
  \begin{beamercolorbox}[wd=.5\paperwidth,ht=2.25ex,dp=1ex,center]{title in head/foot}%
    \usebeamerfont{title in head/foot}\qquad \insertshorttitle \qquad\qquad\qquad \insertframenumber
  \end{beamercolorbox}}%
  \vskip0pt%
}

\graphicspath{./Figures/}

%
\setbeamertemplate{caption}{\raggedright\insertcaption\par}
\usepackage{pdflscape, graphicx, booktabs, dcolumn, listings, amsmath,bbm,courier,pgffor,natbib,enumerate,amssymb,amsfonts,color,hyperref,array,calc,multirow,tikz,amsthm,anyfontsize,bbm, hyperref}

\usepackage[section]{placeins}
\usepackage[english]{babel}
\usepackage[justification=centering]{caption}
\captionsetup{labelformat=empty}
\usepackage[flushleft]{threeparttable}
\newtheorem{findings}[theorem]{Findings}
\usepackage[T1]{fontenc}
\usepackage{lmodern}
\usepackage[percent]{overpic}

\graphicspath{./Figures/}

\newtheorem{prop}{\protect\propositionname}
\addto\captionsamerican{\renewcommand{\propositionname}{Proposition}}
\addto\captionsenglish{\renewcommand{\propositionname}{Proposition}}
\providecommand{\propositionname}{Proposition}
\setbeamertemplate{theorems}[numbered]
\theoremstyle{definition}
\newtheorem*{theorem*}{Definition}

\RequirePackage{ifthen}
\newboolean{sectiontoc}

\AtBeginSection[]{
\begin{frame}[plain]{Outline}
\ifthenelse{\boolean{sectiontoc}}{\tableofcontents[]}{\tableofcontents[currentsection]}
\addtocounter{framenumber}{-1}
\end{frame}}

\newcommand{\firstsection}[1]{\setboolean{sectiontoc}{true} \section{#1} \setboolean{sectiontoc}{false}}
\newcommand{\indep}{\mathrel{\text{\scalebox{1.07}{$\perp\mkern-10mu\perp$}}}}


\setbeamercolor{button}{bg=structure.fg,fg=white}

\begin{document}

\makeatletter
\def\@listi{\leftmargin\leftmarginii \parsep .2em \itemsep 1em}
\def\@listii{\leftmargin\leftmarginii \topsep .2em \parsep .2em \itemsep .2em}
\makeatother

% Title Slide
\begin{frame}[plain]
\titlepage
\addtocounter{framenumber}{-1}
\end{frame}

% Slide 1
\begin{frame}{Walking Through An Example}
\begin{itemize}
    \item We will work through an example together
    \item Let's use the good example of a reproducible paper: 10.1257/pol.20140215
    \item Do each step on your computer
    \item Keep the Wiki open in another tab
\end{itemize}
\end{frame}

%Table of Contents
\begin{frame}{Outline}
\tableofcontents
\end{frame}

% Program context
\section{Access article and download materials}

\begin{frame}{Access article and download materials}
\begin{itemize}
    \item All of the (pre-publication) papers that need to be replicated are listed in the "Replication\_List" sheet on Google Drive 
    \item Follow the URL in the second column to the AEA page
    \item Click on \textbf{Data Set} (under Additional Materials)
    \item Fill out Entry Questionnaire
    \pause
    \item Now you do it
\end{itemize}
\end{frame}

\section{Create and populate repo}

\begin{frame}{Create repo}
\begin{itemize}
    \item Repositories for AEA Verification are on https://bitbucket.org/aeaverification/
    \item Follow the instructions in the Wiki for creating a new repo
    \item \textbf{Today only}: Add your netid to the end of the repo name
    \item Repository name should be "[journal]-[doi]", where
    \begin{itemize}
        \item "[journal]" may be "aej-applied", "aer", "aej-macro" etc.
        \item "[doi]" will be something like "10.1257-pol.20140215"
        \item (replace all spaces and "/" with "-")
    \end{itemize}
    \item All together: "aej-policy-10.1257-pol.20140215-NETID"
    \pause
    \item Now you do it
\end{itemize}
\end{frame}

\begin{frame}{Populate repo}
\begin{itemize}
    \item Follow the instructions in the Wiki for cloning a repo
    \item Save the downloaded materials from the paper in your repo
    \item Now we will use git to add, commit, and push these files to Bitbucket
    \pause
    \item Now you do it
    \item In the terminal, navigate to the repo. Then:
    \begin{enumerate}
        \pause
        \item Use \texttt{git add} to add the appropriate files (careful!)
        \pause
        \item Commit: \texttt{git commit -m "Write your commit message here"}
        \pause
        \item Push: \texttt{git push origin master}
    \end{enumerate}
    \pause
    \item Check Bitbucket--our files should be there now
\end{itemize}
\end{frame}

\begin{frame}{Template looks like this}
\begin{figure}
    \centering
    \includegraphics[scale=0.8]{Figures/template_snip.PNG}
\end{figure}
\end{frame}

\section{Download template}
\begin{frame}{Download template}
\begin{itemize}
    \item Follow the instructions in the Wiki to download the template
\end{itemize}
\end{frame}

\begin{frame}{Template looks like this}
\begin{figure}
    \centering
    \includegraphics[scale=0.8]{Figures/template_snip.PNG}
\end{figure}
\end{frame}

\section{Download template}
\begin{frame}{Download template}
\begin{itemize}
    \item Save only the necessary files in the repo!
    \item Be sure to re-name template-config.do to config.do
    \pause
    \item Now you do it
    \pause
    \item For this example, you should only save: REPLICATION.md, config.do, and .gitignore 
\end{itemize}
\end{frame}

\section{The config file}
\begin{frame}{The config file}
\begin{itemize}
    \item Again: make sure you re-name this!
    \item Add a global basepath and other paths as necessary (the paper's README should help here)
    \begin{figure}
        \centering
        \includegraphics[scale=0.7]{Figures/config_snip.PNG}
    \end{figure}
    \pause
    \item Now you do it
\end{itemize}
\end{frame}

\section{Running main Stata file}
\begin{frame}{Running main Stata file}
\begin{itemize}
    \item Let's go back and re-read the README carefully
    \begin{figure}
        \centering
        \includegraphics[scale=0.8]{Figures/readme_snip.PNG}
    \end{figure}
    \pause
    \item What should we do?
    \pause
    \begin{enumerate}
        \item Create an "Output" directory in the repo and add it to config.do
        \item Add \texttt{include config.do} to the top of the code
    \end{enumerate}
    \pause
    \item What about the note on Table 1?
\end{itemize}
\end{frame}

\section{Committing and pushing to repo}
\begin{frame}{Committing and pushing to repo}
\begin{itemize}
    \item Before running the code, we should \texttt{git commit} again.
    \item After running the code, we will write our report and then \texttt{git commit} and \texttt{git push} again.
    \item Always remember: commit frequently, push (at least) daily
\end{itemize}
\end{frame}




\end{document}