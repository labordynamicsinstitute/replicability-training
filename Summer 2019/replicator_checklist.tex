\documentclass[xcolor={dvipsnames}]{beamer}
\title[Verification Guidance]{What Replicators Will Be Looking For, and Why}
\author{Lars Vilhuber and David Wasser}
\date{April 26, 2019}
\usetheme{Boadilla}
\setbeamertemplate{navigation symbols}{}
\setbeamertemplate{footline}{
  \leavevmode%
  \hbox{%
  \begin{beamercolorbox}[wd=.5\paperwidth,ht=2.25ex,dp=1ex,center]{author in head/foot}%
    \usebeamerfont{author in head/foot}\insertshortauthor
  \end{beamercolorbox}%
  \begin{beamercolorbox}[wd=.5\paperwidth,ht=2.25ex,dp=1ex,center]{title in head/foot}%
    \usebeamerfont{title in head/foot}\qquad \insertshorttitle \qquad\qquad\qquad \insertframenumber
  \end{beamercolorbox}}%
  \vskip0pt%
}

%
\setbeamertemplate{caption}{\raggedright\insertcaption\par}
\usepackage{pdflscape, graphicx, booktabs, dcolumn, listings, amsmath,bbm,courier,pgffor,natbib,enumerate,amssymb,amsfonts,color,hyperref,array,calc,multirow,tikz,amsthm,anyfontsize,bbm}

\usepackage[section]{placeins}
\usepackage[english]{babel}
\usepackage[justification=centering]{caption}
\captionsetup{labelformat=empty}
\usepackage[flushleft]{threeparttable}
\newtheorem{findings}[theorem]{Findings}
\usepackage[T1]{fontenc}
\usepackage{lmodern}
\usepackage[percent]{overpic}

\graphicspath{./Figures/}

\newtheorem{prop}{\protect\propositionname}
\addto\captionsamerican{\renewcommand{\propositionname}{Proposition}}
\addto\captionsenglish{\renewcommand{\propositionname}{Proposition}}
\providecommand{\propositionname}{Proposition}
\setbeamertemplate{theorems}[numbered]
\theoremstyle{definition}
\newtheorem*{theorem*}{Definition}

\RequirePackage{ifthen}
\newboolean{sectiontoc}

\AtBeginSection[]{
\begin{frame}[plain]{Outline}
\ifthenelse{\boolean{sectiontoc}}{\tableofcontents[]}{\tableofcontents[currentsection]}
\addtocounter{framenumber}{-1}
\end{frame}}

\newcommand{\firstsection}[1]{\setboolean{sectiontoc}{true} \section{#1} \setboolean{sectiontoc}{false}}
\newcommand{\indep}{\mathrel{\text{\scalebox{1.07}{$\perp\mkern-10mu\perp$}}}}


\setbeamercolor{button}{bg=structure.fg,fg=white}

\begin{document}

\makeatletter
\def\@listi{\leftmargin\leftmarginii \parsep .2em \itemsep 1em}
\def\@listii{\leftmargin\leftmarginii \topsep .2em \parsep .2em \itemsep .2em}
\makeatother

% Title Slide
\begin{frame}[plain]
\titlepage
\addtocounter{framenumber}{-1}
\end{frame}

% Slide 1
\begin{frame}{Motivation}
\begin{itemize}
    \item We want to be able to verify the work that has been done.
    \item In order to do that, we need some direction.
    \item This document describes:
    \begin{itemize}
        \item What authors should check before providing data and code to journals
        \item What verifier teams should check for in the data and code provided to them for the purpose of verification
    \end{itemize}
\end{itemize}
\end{frame}

\begin{frame}{The README file}
\begin{itemize}
    \item Make sure the contents of the README match the contents of the submission
    \item Cross-check against the paper's data section
    \item Make a list of:
    \begin{itemize}
        \pause 
        \item Data sets referenced or provided
        \pause
        \item Tables
        \item Figures
        \pause 
        \item In-text numbers
    \end{itemize}
    \pause
    \item You can exclude a list of tables, figures, and in-text numbers that appear in an online appendix
\end{itemize}
\end{frame}

\begin{frame}{Documenting data}
\begin{itemize}
    \item For each data source, verify:
    \begin{itemize}
        \item The data set has a clear name
        \item Licensing and access information: accessible to others? right to redistribute?
        \item The data is cited in \textbf{both} the paper and the README
    \end{itemize}
\end{itemize}
\end{frame}

\begin{frame}{Documenting data (continued)}
\begin{itemize}
    \item For each data set, verify:
    \begin{itemize}
        \item All variables are labeled or a codebook is provided
        \item For confidential data, summary statistics are provided \textbf{NOTE FOR LARS: It might be helpful to clarify what we mean by summary stats}
        \item No potentially sensitive information in the data set, including but not limited to:
        \begin{itemize}
            \item Names, SSNs, credit card numbers, addresses etc.
        \end{itemize}
    \end{itemize}
    \item Remember: Someone else is reading your code and using your data. Variable abbreviations that make sense to you might not make any sense to someone else.
\end{itemize}
\end{frame}

\begin{frame}{Documenting tables, figures, and in-text numbers}
\begin{itemize}
    \item Can you identify the part of the code that generates that table, figure, or number?
    \pause
    \item How about the line?
    \pause 
    \begin{itemize}
        \item Add comments to your code and specific references in your README
    \end{itemize}
    \pause
    \item Does the code produce an \textbf{identifiable} output that contains the table, figure, or number?
    \item Remember: Someone else is reading your code and using your data. You might know that Column 1 of Table 2 is created by one piece of code while Column 2 is created by another piece, but they do not.
\end{itemize}
\end{frame}

\begin{frame}{Code verification: where the rubber hits the road}
\begin{itemize}
    \item Create a directory that only contains the provided programs and data
    \item Create a \textit{config} file that stores file paths and relevant system/software parameters in one place
    \item Install all identified requirements
    \begin{itemize}
        \item Easy to forget that \texttt{estout} or \texttt{outreg2} was once manually installed
    \end{itemize}
    \item Run the code using the instructions in the README
    \item Identify all error messages
    \item Identify all outputs as per the README and list of table, figures, and in-text numbers
    \item Compare outputs to those in the paper
\end{itemize}
\end{frame}
\end{document}
