\documentclass[xcolor={dvipsnames}]{beamer}
\title[LDI Training]{Pre-publication Replication Example}
\author{Lars Vilhuber and David Wasser}
\date{August 28, 2019}
\usetheme{Boadilla}
\setbeamertemplate{navigation symbols}{}
\setbeamertemplate{footline}{
  \leavevmode%
  \hbox{%
  \begin{beamercolorbox}[wd=.5\paperwidth,ht=2.25ex,dp=1ex,center]{author in head/foot}%
    \usebeamerfont{author in head/foot}\insertshortauthor
  \end{beamercolorbox}%
  \begin{beamercolorbox}[wd=.5\paperwidth,ht=2.25ex,dp=1ex,center]{title in head/foot}%
    \usebeamerfont{title in head/foot}\qquad \insertshorttitle \qquad\qquad\qquad \insertframenumber
  \end{beamercolorbox}}%
  \vskip0pt%
}

\graphicspath{./Figures/}

%
\setbeamertemplate{caption}{\raggedright\insertcaption\par}
\usepackage{pdflscape, graphicx, booktabs, dcolumn, listings, amsmath,bbm,courier,pgffor,natbib,enumerate,amssymb,amsfonts,color,hyperref,array,calc,multirow,tikz,amsthm,anyfontsize,bbm, hyperref}

\usepackage[section]{placeins}
\usepackage[english]{babel}
\usepackage[justification=centering]{caption}
\captionsetup{labelformat=empty}
\usepackage[flushleft]{threeparttable}
\newtheorem{findings}[theorem]{Findings}
\usepackage[T1]{fontenc}
\usepackage{lmodern}
\usepackage[percent]{overpic}

\graphicspath{./Figures/}

\newtheorem{prop}{\protect\propositionname}
\addto\captionsamerican{\renewcommand{\propositionname}{Proposition}}
\addto\captionsenglish{\renewcommand{\propositionname}{Proposition}}
\providecommand{\propositionname}{Proposition}
\setbeamertemplate{theorems}[numbered]
\theoremstyle{definition}
\newtheorem*{theorem*}{Definition}

\RequirePackage{ifthen}
\newboolean{sectiontoc}

\AtBeginSection[]{
\begin{frame}[plain]{Outline}
\ifthenelse{\boolean{sectiontoc}}{\tableofcontents[]}{\tableofcontents[currentsection]}
\addtocounter{framenumber}{-1}
\end{frame}}

\newcommand{\firstsection}[1]{\setboolean{sectiontoc}{true} \section{#1} \setboolean{sectiontoc}{false}}
\newcommand{\indep}{\mathrel{\text{\scalebox{1.07}{$\perp\mkern-10mu\perp$}}}}


\setbeamercolor{button}{bg=structure.fg,fg=white}

\begin{document}

\makeatletter
\def\@listi{\leftmargin\leftmarginii \parsep .2em \itemsep 1em}
\def\@listii{\leftmargin\leftmarginii \topsep .2em \parsep .2em \itemsep .2em}
\makeatother

% Title Slide
\begin{frame}[plain]
\titlepage
\addtocounter{framenumber}{-1}
\end{frame}

% Slide 1
\begin{frame}{Walking Through An Example}
\begin{itemize}
    \item We will work through an example of a pre-publication replication together
    \item We will use a replication package that has already been published on openICPSR: \#110803
    \item You will need to set up an account on openICPSR
    \item Do each step on your computer
    \item Keep the Wiki open in another tab
    \item \textbf{We will not use Jira for this example, but Jira directions are included for your reference}
\end{itemize}
\end{frame}

%Table of Contents
\begin{frame}{Outline}
\tableofcontents
\end{frame}

% Program context
\section{Access article and download materials}

\begin{frame}{Access article and download materials}
\begin{itemize}
    \item Find the issue on Jira and advance it from \textbf{Open} to \textbf{In Progress}
    \item The pdf of the manuscript will be attached to the Jira issue
    \item The replication materials are on openICPSR: log in and search for the project number on the Find Data page
    \begin{itemize}
        \item On the real Jira ticket, the project URL will already be in the Code Provenance field
    \end{itemize}
    \item You should see a project named ``Data and Code for Uncertainty and Business Cycles Replication File''
    \item On the righthand side, click \textbf{Download This Project}
    \item Fill out Entry Questionnaire
\end{itemize}
\end{frame}

\section{Create and populate repo}

\begin{frame}{Create repo}
\begin{itemize}
    \item Repositories for AEA Verification are on https://bitbucket.org/aeaverification/
    \item Follow the instructions in the Wiki for creating a new repo
    \item \textbf{Today only}: Add your netid to the end of the repo name
    \item Repository name should be the same as the Jira ticket number, e.g. AEAREP-14
    \item \textbf{Today only}: Make the repo name ``TEST-netid''
\end{itemize}
\end{frame}

\begin{frame}{Populate repo}
\begin{itemize}
    \item Follow the instructions in the Wiki for cloning a repo
    \item Save the downloaded materials from the paper in your repo
    \item Now we will use git to \texttt{add}, \texttt{commit}, and \texttt{push} these files to Bitbucket
    \item In the terminal, navigate to the repo. Then:
    \begin{enumerate}
        \item Use \texttt{git add} to add the appropriate files (careful!)
        \pause
        \item Commit: \texttt{git commit -m "\textit{REPO NAME} \#comment Write your commit message here"}
        \begin{itemize}
            \item This is ``smart commit'' that syncs Jira and Bitbucket
        \end{itemize}
        \pause
        \item Push: \texttt{git push origin master}
    \end{enumerate}
    \pause
    \item Check Bitbucket--our files should be there now
\end{itemize}
\end{frame}

\section{Download template}
\begin{frame}{Download template}
\begin{itemize}
    \item Follow the instructions in the Wiki to download the template
\end{itemize}
\end{frame}

\begin{frame}{Template looks like this}
\begin{figure}
    \centering
    \includegraphics[scale=0.8]{Figures/template_snip.PNG}
\end{figure}
\end{frame}

\section{Download template}
\begin{frame}{Download template}
\begin{itemize}
    \item Save only the necessary files in the repo!
    \item For this example, you should only save: REPLICATION.md, .gitignore, and SRC
    \item This replication is in Matlab, but most of the replications you do will be in Stata
    \item You can follow the post-publication example for Stata-specific help from this point forward
    \item The most important difference between the Stata process and the Matlab process is the confog.do file in Stata (see the post-publication example for more on this)
\end{itemize}
\end{frame}

\section{Jira Process}
\begin{frame}{Jira Process}
\begin{itemize}
    \item Fill out the Entry Questionnaire, advance Jira to Code
    \item Fill out code provenance (if it has not been filled out already), advance Jira to Data
    \item Fill out location of data--typically the openICPSR project number, advance Jira to Verification
\end{itemize}
\end{frame}

\section{Verification}
\begin{frame}{Verification}
\begin{itemize}
    \item Let's try to replicate Figure 1 today
    \item Log into Ciser
    \item Create a directory in Documents for this example called ``jira\_example''
    \item Use \texttt{git clone} to get your Bitbucket repo and its contents onto Ciser
    \item Open gen\_figure1.m in Matlab and run it
    \item What do you find?
\end{itemize}
\end{frame}

\section{Committing and pushing to repo}
\begin{frame}{Committing and pushing to repo}
\begin{itemize}
    \item After running the code, we will write our report and then \texttt{git commit} and \texttt{git push} again.
    \item Always remember: commit frequently, push (at least) daily
\end{itemize}
\end{frame}
\end{document}